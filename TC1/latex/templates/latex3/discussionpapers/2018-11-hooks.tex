\documentclass{article}


\title{Thoughts on Hooks in code}
\author{Frank Mittelbach \and
%                        \and  
%                        \and  
  \LaTeX{} Project Team}

\date{2018-11-29}

\newcounter{hook}

\newenvironment{hook}[3][new]  % new/exists, area, name 
               {\refstepcounter{hook}%
                 \begin{itemize}\item[\textbf{(\thehook)}]\textbf{#3} (#2/\textit{#1})}
               {\end{itemize}}

\newcommand\cs[1]{\texttt{\textbackslash #1}}               
\newcommand\pkg[1]{\texttt{#1}}               

\begin{document}

\maketitle

\tableofcontents

\section{Introduction}

This document collects some thoughts on places for hooks (i.e., code
interfaces where other code can add material for execution. Collected
are existing \LaTeXe{} hooks (both by the kernel and/or by package) as
well a hooks that do not exist but would be beneficial to have for one
or the other reason. As of now it mainly discusses the \LaTeXe{}
situation.

For each hook we document
\begin{itemize}
\item a ``name'';
\item the area where it applies;
\item whether it is ``new'' or does already exist (if so where);
\item and the major reason why it would be beneficial.
\end{itemize}
If there are several distinct use cases each could be added using
\cs{item}. You can cross reference hooks using
\cs{label}/\cs{ref}.



\section{Document structure}

\begin{hook}[kernel]{doc-structure}{begin-document}
\end{hook}

\begin{hook}[kernel]{doc-structure}{end-document}
\end{hook}


\section{Heading commands}

\begin{hook}{heading}{before-break}
\item
  For heading commands it is helpful if one can issue \cs{mark}
  commands, e.g. \cs{PutMark} from \pkg{xmarks} to support running
  header or footer setup. This has to be possible directly after
  heading title (where it is currently supported through commands like
  \cs{sectionmark}) but also directly before the break to determine
  the placement of the heading (e.g., at the very top of a page/column
  or elsewhere). The mark needs to come after the heading has done its
  magic with any preceding space, it can't hide such space from the
  command. In other words it need to happen basically in the middle of
  \cs{addpenalty}.
\item
  For L3 the galley concept might be able to handle this properly, but
  most likely also need more than one point of code injection.
\end{hook}


\section{Output routine}

\subsection{Making columns}

\begin{hook}{\cs{@makecol}}{pre}
\item
  \pkg{manyfoot} and similar packages like to take control at the very
  beginning of column generation.
\end{hook}

\begin{hook}{\cs{@makecol}}{post}
\item
  And we may have some post-processing going on after the column is
  made up.
\end{hook}

\begin{hook}{\cs{@makecol}}{prepare-footins}
\item
  To support manipulation of footnote text (like footnotes as paras,
  in 2 columns etc).

  The assembled footnotes are inside box \cs{footins} and any
  manipulation needs to globally write them back in there.
\end{hook}



%\section{}

\end{document}
