
% \iffalse
%<*driver>
\documentclass[full]{l3doc}
\begin{document}
\DocInput{\jobname.dtx}
\clearpage
\PrintIndex
\end{document}
%</driver>
% \fi
%
% \title{Example l3doc document}
% \author{The \LaTeX3 Project}
%
% \maketitle
%
% \begin{documentation}
%
% \section{Documentation}
%
% \begin{function}{\ProcessKeysOptions}
%   \begin{syntax}
%     \cs{ProcessKeysOptions} \Arg{module}
%   \end{syntax}
%   This function is used to check the current
%   option list against the keys defined for \Arg{module}. Global (class)
%   options and local (package) options are checked when this function
%   is called in a package. Each option which does match a key name is
%   then used to attempt to set the appropriate key using
%   \cs{keys_set:nn}. For example, the option defined earlier would be
%   processed by the line
%   \begin{verbatim}
%     \ProcessKeysOptions { module }
%   \end{verbatim}
% \end{function}
%
% \begin{function}{\l_@@_latexe_options_clist}
%   An internal variable.
%   Here is a reference to \cs{bool_set_false:N}.
% \end{function}
%
%\end{documentation}
%
% \clearpage
%
%\begin{implementation}
%
%\section{Implementation}
%
%    \begin{macrocode}
%<*package>
%    \end{macrocode}
%
%    \begin{macrocode}
\ProvidesExplPackage{l3keys2e}{2017/09/18}{}
  {LaTeX2e option processing using LaTeX3 keys}
%    \end{macrocode}
%
% Non-standard variants.
%    \begin{macrocode}
\cs_generate_variant:Nn \clist_put_right:Nn { Nv }
\cs_generate_variant:Nn \keys_if_exist:nnT  { nx }
\cs_generate_variant:Nn \keys_if_exist:nnTF { nx }
%    \end{macrocode}
%
% \begin{macro}{\l_@@_latexe_options_clist}
%   A single list is used for all options, into which they are collected
%   before processing.
%    \begin{macrocode}
\clist_new:N \l_@@_latexe_options_clist
%    \end{macrocode}
% \end{macro}
%
% \begin{macro}{\l_@@_process_class_bool}
%   A flag to indicate that class options should be processed for
%   packages.
%    \begin{macrocode}
\bool_new:N \l_@@_process_class_bool
%    \end{macrocode}
% \end{macro}
%
% \begin{macro}{\ProcessKeysOptions}
% \begin{macro}{\ProcessKeysPackageOptions}
%   The user macro are simply wrappers around the internal process. In
%   contrast to other similar packages, the module name is always required
%   here.
%    \begin{macrocode}
\cs_new_protected:Npn \ProcessKeysOptions #1
  {
    \bool_set_true:N \l_@@_process_class_bool
    \@@_latexe_options:n {#1}
  }
\cs_new_protected:Npn \ProcessKeysPackageOptions #1
  {
    \bool_set_false:N \l_@@_process_class_bool
    \@@_latexe_options:n {#1}
  }
\@onlypreamble \ProcessKeysOptions
\@onlypreamble \ProcessKeysPackageOptions
%    \end{macrocode}
%\end{macro}
%\end{macro}
%
%
%    \begin{macrocode}
%</package>
%    \end{macrocode}
%
% \end{implementation}
%
%
\endinput