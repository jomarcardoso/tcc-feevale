\documentclass[
	% -- opções da classe memoir --
	12pt,				% tamanho da fonte
 	oneside,			% para impressão em recto e verso. Oposto a oneside
 	openany,
	a4paper,			% tamanho do papel.
	% -- opções da classe abntex2 --
	%chapter=TITLE,		% títulos de capítulos convertidos em letras maiúsculas
	%section=TITLE,		% títulos de seções convertidos em letras maiúsculas
	%subsection=TITLE,	% títulos de subseções convertidos em letras maiúsculas
	%subsubsection=TITLE,% títulos de subsubseções convertidos em letras maiúsculas
	% -- opções do pacote babel --
	english,			% idioma adicional para hifenização
	french,				% idioma adicional para hifenização
	spanish,			% idioma adicional para hifenização
	brazil				% o último idioma é o principal do documento
	]{abntex2}

% ---
% Pacotes básicos
% ---
\usepackage{lmodern}			% Usa a fonte Latin Modern
\usepackage[T1]{fontenc}		% Selecao de codigos de fonte.
\usepackage[utf8]{inputenc}		% Codificacao do documento (conversão automática dos acentos)
\usepackage{indentfirst}		% Indenta o primeiro parágrafo de cada seção.
\usepackage{color}				% Controle das cores
\usepackage{graphicx}			% Inclusão de gráficos
\usepackage{microtype} 			% para melhorias de justificação
% ---

% ---
% Pacotes de citações
% ---
\usepackage[brazilian,hyperpageref]{backref}	 % Paginas com as citações na bibl
\usepackage[alf]{abntex2cite}	% Citações padrão ABNT

% ---
% Configurações do pacote backref
% Usado sem a opção hyperpageref de backref
\renewcommand{\backrefpagesname}{Citado na(s) página(s):~}
% Texto padrão antes do número das páginas
\renewcommand{\backref}{}
% Define os textos da citação
\renewcommand*{\backrefalt}[4]{
	\ifcase #1 %
	Nenhuma citação no texto.%
	\or
	Citado na página #2.%
	\else
	Citado #1 vezes nas páginas #2.%
	\fi}%
% ---

% ---
% Informações de dados para CAPA e FOLHA DE ROSTO
% ---
\titulo {
	FRAMEWORK PARA CRIAÇÃO DE COMPONENTES REUTILIZÁVEIS
}
\autor{Jomar Antônio Cardoso}
\local{Novo Hamburgo}
\data{2018}
\orientador{Orientador...}
\instituicao{%
	Universidade Feevale
	\par
	Sistemas de Informação
	\par
	Teste de Conclusão 1}
\tipotrabalho{Anteprojeto de Trabalho de Conclusão}
% O preambulo deve conter o tipo do trabalho, o objetivo,
% o nome da instituição e a área de concentração
\preambulo{Anteprojeto de Trabalho de Conclusão de
	Curso, apresentado como requisito parcial à obtenção do grau de Bacharel em Sistemas de Informação pela Universidade Feevale}
% ---

% ---
% Configurações de aparência do PDF final

% alterando o aspecto da cor azul
\definecolor{blue}{RGB}{41,5,195}

% informações do PDF
\makeatletter
\hypersetup{
	%pagebackref=true,
	pdftitle={\@title},
	pdfauthor={\@author},
	pdfsubject={\imprimirpreambulo},
	pdfcreator={LaTeX with abnTeX2},
	pdfkeywords={abnt}{latex}{abntex}{abntex2}{trabalho acadêmico},
	colorlinks=true,       		% false: boxed links; true: colored links
	linkcolor=blue,          	% color of internal links
	citecolor=blue,        		% color of links to bibliography
	filecolor=magenta,      		% color of file links
	urlcolor=blue,
	bookmarksdepth=4
}
\makeatother
% ---

% ---
% Posiciona figuras e tabelas no topo da página quando adicionadas sozinhas
% em um página em branco. Ver https://github.com/abntex/abntex2/issues/170
\makeatletter
\setlength{\@fptop}{5pt} % Set distance from top of page to first float
\makeatother
% ---

% ---
% Possibilita criação de Quadros e Lista de quadros.
% Ver https://github.com/abntex/abntex2/issues/176
%
\newcommand{\quadroname}{Quadro}
\newcommand{\listofquadrosname}{Lista de quadros}

\newfloat[chapter]{quadro}{loq}{\quadroname}
\newlistof{listofquadros}{loq}{\listofquadrosname}
\newlistentry{quadro}{loq}{0}

% configurações para atender às regras da ABNT
\setfloatadjustment{quadro}{\centering}
\counterwithout{quadro}{chapter}
\renewcommand{\cftquadroname}{\quadroname\space}
\renewcommand*{\cftquadroaftersnum}{\hfill--\hfill}

\setfloatlocations{quadro}{hbtp} % Ver https://github.com/abntex/abntex2/issues/176
% ---

% ---
% Espaçamentos entre linhas e parágrafos
% ---

% O tamanho do parágrafo é dado por:
\setlength{\parindent}{1.3cm}

% Controle do espaçamento entre um parágrafo e outro:
\setlength{\parskip}{0.2cm}  % tente também \onelineskip

% ---
% compila o indice
% ---
\makeindex
% ---

% ----
% Início do documento
% ----

\begin{document}

% Seleciona o idioma do documento (conforme pacotes do babel)
%\selectlanguage{english}
\selectlanguage{brazil}

% Retira espaço extra obsoleto entre as frases.
\frenchspacing

% ----------------------------------------------------------
% ELEMENTOS PRÉ-TEXTUAIS
% ----------------------------------------------------------
\pretextual
\imprimircapa
\imprimirfolhaderosto*


% resumo em português
\setlength{\absparsep}{18pt} % ajusta o espaçamento dos parágrafos do resumo
\begin{resumo}
Este trabalho é uma proposta de um framework de padrões de projeto para desenvolvimento de componentes reutilizáveis no paradigma funcional. Faz-se necessário a identificação de, todas as necessidades para a criação de um componente, todas as dificuldades para a execução da mesma, quais padrões de projeto devam ser usados. Sendo assim, este tem como objetivo fornecer um framework que atenda os requisitos citados, que formalize a criação de componentes do paradigma funcional.

\textbf{Palavras-chave}: Framework. Padrões de projeto.
\end{resumo}

% ---
% inserir o sumario
% ---
\pdfbookmark[0]{\contentsname}{toc}
\tableofcontents*
\cleardoublepage
% ---

% ----------------------------------------------------------
% ELEMENTOS TEXTUAIS
% ----------------------------------------------------------
\textual

\chapter{Motivação}

Este trabalho é uma análise das necessidades para a criação de componentes reutilizáveis, também é um estudo dos vários padrões de projeto e como eles podem contribuir.

Como desenvolvedor frequentemente me deparo com a situação de ter que criar uma nova funcionalidade e sempre penso se há como esta ser um componente totalmente reutilizável e então desenvolvo seguindo o que acredito ser o código mais independente e genérico possível. A primeira utilização deste componente funciona perfeitamente, obviamente, pois foi criado para atender uma necessidade específica. 

Os problemas aparecem geralmente de três formas, quando precisa-se reutilizar o componente em outro contexto e a solução não serve, quando alguma outra pessoa precisa implementar o componente e não sabe como utilizá-lo e quando a funcionalidade inicial requisitada é modificada ou adicionado algo a mais e o componente tem de ser replicado ou refeito para atender especificamente a nova necessidade.

Como \cite[p.~XIV]{aPatternLanguage} mencionam, cada padrão descreve um problema que ocorre repetidas vezes em nosso meio ambiente e então descreve o ponto central da solução do problema, de modo que você possa usar a mesma solução milhares de vezes, mas sem jamais ter de repeti-la. Então temos que cada padrão possui um problema e uma solução, e que esta deva servir sempre para aquele mesmo problema. No caso dos componentes e das funcionalidades citadas acima, são as soluções e os problemas de um padrão.

Segundo ... aplicação é um conjunto de soluções(citar) e seguindo o pensamento acima segundo \cite[p.~XVI]{aPatternLanguage} Nenhum padrão é uma entidade isolada. Cada padrão existe somente porque é sustentado por outros padrões: os padrões maiores, dentro dos quais ele se inclui, os padrões do mesmo tamanho, que o circundam, e os padrões menores, nele inserido. Então se um sistema é um conjunto de componentes que solucionam problemas, este também pode ser considerado um padrão maior que é sustentado pelos componentes.


\begin{itemize}
	 \item Administradores possam cadastrar todos os agentes envolvidos, 
	 motoristas, passageiros, instituições, jornadas, veículos e administradores.

	\item Administradores criarem listas de chamada.

	\item Simplicidade. Foi identificado (referenciar pesquisa
	feita pela Kangur), que \% dos possíveis usuários do sistema 
	são leigos em relação a sistemas de informação. Então é 
	solicitado que a solução contenha a menor quantidade possível 
	de caminhos e escolhas para realização de cada tarefa.

	\item Motoristas possam ter as informações dos endereços dos 
	passageiros de suas respectivas jornadas.

	\item Possibilitar aos passageiros o cancelamento de sua presença numa jornada.
\end{itemize}

Segundo \cite[p.~98]{spinolapessoa1998}, um “Sistema de Informação (S.I.) é um
sistema que cria um ambiente integrado e consistente, capaz de fornecer as informações necessárias a todos os usuários”, então além da solicitação do cliente é preciso verificar se todos os usuários do sistema terão acesso as informações que apetece-lhes, também que a execução deste torne-o confiável e acessível.

É de grande importância o planejamento dos Sistemas de Informação, pois definem o futuro dos sistemas da organização. No caso deste sistema do estudo se trata de um inicial com poucas funcionalidades, mas com possível potencial de expansão, então deve-se ser feito também uma análise de onde será o crescimento e como preparar o sistema atual para receber os novos módulos.


É identificado que o tipo de sistema se trata de um Sistemas de informações transacionais (SIT)

De acordo com \cite{obrien2004} São as informações rotineiras efetuadas. Essas informações normalmente alimentam um banco de dados para futuras consultas. 

\cite[p.~6]{obrien2004} “é um conjunto organizado de pessoas, hardware, software, redes de comunicações e recursos de dados que coleta, transforma e dissemina informações em uma organização”.

\chapter{Objetivos}

Criar uma modelagem para desenvolvimento de um sistema de informação capaz de
atender de forma eficiente as necessidades do cliente.

\begin{itemize}
	\item Identificar todas as informações e todos os agentes.
	
	\item Entender as necessidades do solicitante.
	
	\item Prever os possíveis crescimentos do sistema.
	
	\item Apresentar a modelagem.
\end{itemize}

\chapter{Metodologia}

Para auxiliar na realização deste projeto pretendo me basear na leitura dos principais livros do assunto, também analisar casos de usos e demais estudos semelhantes.

\postextual
\bibliography{anteprojeto-references}

\end{document}
